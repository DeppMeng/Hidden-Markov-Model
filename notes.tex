% last updated in April 2002 by Antje Endemann
% Based on CVPR 07 and LNCS, with modifications by DAF, AZ and elle, 2008 and AA, 2010, and CC, 2011; TT, 2014; AAS, 2016

\documentclass[runningheads]{llncs}
\usepackage{graphicx}
\usepackage{amsmath,amssymb} % define this before the line numbering.
\usepackage{ruler}
\usepackage{color}
\usepackage[width=122mm,left=12mm,paperwidth=146mm,height=193mm,top=12mm,paperheight=217mm]{geometry}
\usepackage{multirow, multicol}
    
% \newtheorem{theorem}{Theorem}

\begin{document}
% \renewcommand\thelinenumber{\color[rgb]{0.2,0.5,0.8}\normalfont\sffamily\scriptsize\arabic{linenumber}\color[rgb]{0,0,0}}
% \renewcommand\makeLineNumber {\hss\thelinenumber\ \hspace{6mm} \rlap{\hskip\textwidth\ \hspace{6.5mm}\thelinenumber}}
% \linenumbers
\pagestyle{headings}
\mainmatter
\def\ECCV18SubNumber{1452}  % Insert your submission number here

\title{Hidden Markov Model} % Replace with your title

\titlerunning{ECCV-18 submission ID \ECCV18SubNumber}

\authorrunning{ECCV-18 submission ID \ECCV18SubNumber}

\author{Depu Meng}
\institute{Oct. $2018$}


\maketitle
\section{Basic Concepts \& Examples}
\subsection{Basic Concepts}
\paragraph{Definition 1.1}
Assume $X = \{ X_n; n \geq 1 \} $ is a Markov chain over finite state space
$\Phi = \{ 1, 2, ...,K \} $, if states of $X$ are unobservable,
$Y = \{ Y_n; n \geq 1 \} $ is an observable random variable series over
finite set $V = \{ v_1, v_2,...,v_L \} $ that is correlated with $X$,
then $(X, Y) = \{ (X_n, Y_n); n \geq 1 \} $ is a \emph{Hidden Markov Chain}.
\par
Denote $\pi = \{ \pi_1, \pi_2,...,\pi_k \} $ is the initial distribution of $X$,
$A = [a_{ij}]$ is the transition matrix of $X$
\begin{align}
    a_{ij} = P\{X_{n+1} = j | X_n = i\}, i, j \in \Phi
\end{align}
is the one-step transition probality of X. Denote
\begin{align}
    b_{ij} = P\{ Y_n = v_j | X_n = i \}, i \in \Phi, v_j \in V
\end{align}
represents the probability that $Y$ equals to $v_j$ given $X$ is at state $i$ at time $n$.
If $X$ is homogeneous, then $Y$ is also irrelavent with time $n$.
Denote $B = [b_{ij}]$ as the observation probability matrix.
Due to the unobservability of $X$, $\pi, A, B$ can not be directly measured.
Generally, we call parameter set $\lambda = \{ \pi, A, B \} $
the math model of Hidden Markov chain $(X, Y)$, as well as \emph{Hidden Markov Model} (HMM).
\paragraph{Property 1.1}
Assume $N$ is the length of time on observation,
denote $\mathbf{X}= \{ X_1, X_2,..., X_N \} $, $\mathbf{Y} = \{ Y_1, Y_2,..., Y_N \} $
as the sample series of Markov chain $X$ and observation series $Y$ in the time period $1\approx N$
respectively, then the joint distribution of $\mathbf{X}$ and $\mathbf{Y}$ satisfy
the following \emph{Hidden Markov Condition}.
\begin{align}
    P\{ \mathbf{X} = x, \mathbf{Y} = y \} = \pi_{i_1} b_{i_1 j_1}... b_{i_{N-1} j_{N-1}} b_{i_{N-1} j_{N}} b_{i_{N} j_{N}},
\end{align}
where $x = \{ i_1, ..., i_N \} $, $y = \{ v_{j_1}, ..., v_{j_N} \} $.

\subsection{Examples}
\paragraph{Coin flipping problem}
Stochasticly choose a coin from two coins indexed $1 and 2$, and then toss it,
observe the result and repeat this process.
Denote $X_n$ as the coin chosen in $t^{th}$ time,
then $X = \{ X_n; n \geq 1 \} $ is a Markov chain over state space
$\Phi = \{ 1, 2 \} $.
Denote $Y_n$ as the result of $n^{th}$ experiment,
then $Y = \{ Y_n; n \geq 1\} $ is observation series
that takes values from $V =  \{ H, T \} $, where $H$ means the front side and $T$
means the back side.
State transition matrix and observation probability matrix are
\begin{align}
    A =
    \begin{bmatrix}
        a_{11} &a_{12} \\
        a_{21} &a_{22} \\
    \end{bmatrix},
    B =
    \begin{bmatrix}
        b_{11} &b_{12} \\
        b_{21} &b_{22} \\
    \end{bmatrix},
\end{align}
where summation of each row equals to $1$.
For initial distribution $\pi = \{ \pi_1, \pi_2 \} $,
we have $\pi_1 + \pi_2 = 1$.
Then $\lambda  =\{ \pi, A, B \} $ is the corresponding HMM of Hidden Markov chain $(X, Y)$,
which contains $5$ unknown parameters.

\section{Basic Problems \& Solutions}
\subsection{Basic Problems}
\paragraph{Problem 1}
For a specific observation sample series $\mathbf{Y} = \{ Y_1, ..., Y_n \} $,
known it is generated by one of the given HMMs, determine which model that generates the 
sample series is called \emph{Pattern Recognition} problem, as well as \emph{classification} problem.
\paragraph{Problem 2}
From a series of observation samples $\mathbf{Y} = \{ Y_1, ..., Y_n \} $ and known HMM
$\lambda = \{ \pi, A, B \} $, give the best estimate of hidden states is called \emph{State Estimate}
problem, as well as \emph{Decoding} problem.
\paragraph{Problem 3}
From a series of observation samples $\mathbf{Y} = \{ Y_1, ..., Y_n \} $,
give the best estimate of parameter set $\lambda = \{ \pi, A, B \} $,
we call it \emph{Model Learning} problem.

\end{document}