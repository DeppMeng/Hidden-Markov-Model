\documentclass[runningheads]{llncs}
    \usepackage{graphicx}
    \usepackage{amsmath,amssymb} % define this before the line numbering.
    \usepackage{ruler}
    \usepackage{color}
    \usepackage[width=122mm,left=12mm,paperwidth=146mm,height=193mm,top=12mm,paperheight=217mm]{geometry}
    \usepackage{multirow, multicol}
        
    % \newtheorem{theorem}{Theorem}[section]
    % \newtheorem{lemma}[theorem]{Lemma}
    % \newtheorem{definition}{Definition}[section]
    
    \begin{document}
    % \renewcommand\thelinenumber{\color[rgb]{0.2,0.5,0.8}\normalfont\sffamily\scriptsize\arabic{linenumber}\color[rgb]{0,0,0}}
    % \renewcommand\makeLineNumber {\hss\thelinenumber\ \hspace{6mm} \rlap{\hskip\textwidth\ \hspace{6.5mm}\thelinenumber}}
    % \linenumbers
    \pagestyle{headings}
    \mainmatter
    \def\ECCV18SubNumber{1452}  % Insert your submission number here
    
    \title{Homework} % Replace with your title
    
    \titlerunning{ECCV-18 submission ID \ECCV18SubNumber}
    
    \authorrunning{ECCV-18 submission ID \ECCV18SubNumber}
    
    \author{Depu Meng}
    \institute{Oct. $2018$}
    
    
    \maketitle
    \subsection{}
    \begin{proof}
        (1). From the definition of measure, we have
        \begin{equation}
            \mu (\bigcup_{n=1}^\infty E_n) = \sum_{n=1}^\infty \mu(E_n)
        \end{equation}
        if $E_i \bigcap E_j = \varnothing, i \neq j$.
        So for any $E \in \mathcal{F}$,
        \begin{equation}
            \mu(E) = \mu ( \varnothing \bigcup E) = \mu(E) + \mu(\varnothing)
        \end{equation}
        \par
        that is, $\mu(\varnothing) = 0$.
        \par
        (2).Consider $E_1, E_2,..., E_n, E_{n+1}, ... \in \mathcal{F}$,
        $E_i \bigcap E_j = \varnothing, i \neq j$, then from the definition
        we have
        \begin{align}
            \mu (\bigcup_{k=1}^\infty E_k) &= \sum_{k=1}^\infty \mu(E_k) \\
            \mu (\bigcup_{k=n+1}^\infty E_k) &= \sum_{k=n+1}^\infty \mu(E_k)
        \end{align}
        so that we have
        \begin{align}
            \mu (\bigcup_{k=1}^n E_k) &= \sum_{k=1}^n \mu(E_k)
        \end{align}
        \par
        (3). If $E_1 \subset E_2$, then $E_2 = E_1 \bigcup (E_2 - E_1)$,
        apparently $E_1 \bigcap (E_2 - E_1) = \varnothing$.
        Then from the definition we have
        \begin{align}
            \mu(E_2) = \mu ( E_1 \bigcup (E_2 - E_1)) = \mu(E_1) + \mu(E_2 - E_1) \geq \mu(E_1)
        \end{align}
    \end{proof}
    \subsection{}
    \begin{proof}
        \begin{align}
            P(A) &= P(A \bigcap (\bigcup_{i=1}^\infty B_i)) 
            = P(\bigcup_{i=1}^\infty (A \bigcap B_i)) \\
            &= \sum_{i=1}^\infty P(A \bigcap B_i)
            = \sum_{i=1}^\infty P(B_i) P(A | B_i)
        \end{align}
    \end{proof}
    \subsection{}
    \begin{proof}
        \begin{align}
            \int_0^\infty (1 - F(t))dt &= \int_0^\infty (1 - P\{ X \leq t \})dt \\
            &= \int_0^\infty P \{ X > t \}dt \\
            &= \int_0^\infty \int_t^\infty f(s)dsdt \\
            &= \int_0^\infty \int_0^s f(s) dt ds \\
            &= \int_0^\infty s f(s) ds \\
            &= E[X]
        \end{align}
    \end{proof}
    \subsection{}
    \begin{proof}
        From property (2), we have
        \begin{align}
            P \{ N(s + t) - N(s) = k \} = P \{ N(t) = k \}
        \end{align}
        Firstly consider $k = 0$, denote $P_k(t) = P \{ N(t) = k \}$,
        apparently for $h > 0$
        \begin{align}
            P \{ N(t + h) = 0 \} &= P \{ N(t) = 0, N(t + h) - N(t) = 0 \} \\
            &= P_0(t)P_0(h)
        \end{align}
        On the other hand, from property (3), (4), we have
        \begin{align}
            P_0(h) = 1 - (\lambda h + o(h))
        \end{align}
        So that
        \begin{align}
            \frac{P_0(t + h) - P_0(t)}{h} = -(\lambda P_0(t) + \frac{o(h)}{h})
        \end{align}
        Let $h \rightarrow 0$,
        \begin{align}
            P_0'(t) = - \lambda P_0(t)
        \end{align}
        By solving this differential equation with constraint $P_0(0) = 1$,
        we can get
        \begin{align}
            P_0(t) = e^{-\lambda t}
        \end{align}
    \end{proof}


    \end{document}