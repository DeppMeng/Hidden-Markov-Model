% last updated in April 2002 by Antje Endemann
% Based on CVPR 07 and LNCS, with modifications by DAF, AZ and elle, 2008 and AA, 2010, and CC, 2011; TT, 2014; AAS, 2016

\documentclass[runningheads]{llncs}
    \usepackage{graphicx}
    \usepackage{amsmath,amssymb} % define this before the line numbering.
    \usepackage{ruler}
    \usepackage{color}
    \usepackage[width=122mm,left=12mm,paperwidth=146mm,height=193mm,top=12mm,paperheight=217mm]{geometry}
    \usepackage{multirow, multicol}
        
    % \newtheorem{theorem}{Theorem}
    
    \begin{document}
    % \renewcommand\thelinenumber{\color[rgb]{0.2,0.5,0.8}\normalfont\sffamily\scriptsize\arabic{linenumber}\color[rgb]{0,0,0}}
    % \renewcommand\makeLineNumber {\hss\thelinenumber\ \hspace{6mm} \rlap{\hskip\textwidth\ \hspace{6.5mm}\thelinenumber}}
    % \linenumbers
    \pagestyle{headings}
    \mainmatter
    \def\ECCV18SubNumber{1452}  % Insert your submission number here
    
    \title{Markov Decision Process} % Replace with your title
    
    \titlerunning{ECCV-18 submission ID \ECCV18SubNumber}
    
    \authorrunning{ECCV-18 submission ID \ECCV18SubNumber}
    
    \author{Depu Meng}
    \institute{Oct. $2018$}
    
    
    \maketitle
    \section{Basic Concepts \& Examples}
    \emph{Markov Decision Process} (MDP) is a Markov Process that decisions are involved.
    Generally, an MDP can described by a quintuple:
    \begin{itemize}
        \item A Markov Process or an Extended Markov Process to describe the process.
        \item A state space.
        \item An action space.
        \item A state transition function.
        \item A performance function.
    \end{itemize}

    MDP can be divided into Continuous-Time MDP and Discrete-Time MDP
    according to the time factor of the Markov Process;
    MDP can also be divided into MDP and Semi-MDP and Partially-Observable MDP.

    \subsection{Policy and policy space}
    In this section, we will take policy and policy space in DTMDP as an example.
    \par
    A DTMDP quintuple can be denoted as $\{ X, \Phi, A, P_{ij}(a), f(i, a) \}$,
    \par\noindent
    $X = \{ X_n; n \geq 0 \}$ is a Discrete-Time Markov Process, $\Phi = \{i \}$
    and $A = \{ a \}$ are state space and action space of this process respectively.
    For $P_{ij}(a)$, appearently we have $P_{ij}(a) \geq 0$ and
    $\sum_{j \in \Phi} P_{ij}(a) = 1$.
    A DTMDP sample orbit can be described as $\{ i_0, a_0, i_1, a_1,...\}$.
    Denote $h_n = \{ i_0, a_0,...,i_{n-1}, a_{n-1}, i_n \}$ as the history
    before time $n$.
    \par
    A general policy is defined as
    \begin{align}
        v = (v_0(a | h_0), v_1(a | h_1),...)    
    \end{align}
    Infact, a general policy is a series of action defined on decision time,
    which is also a stochastic policy if not specified.
    The set that contains all policies like (1) is a policy space, denoted as $\Pi$.
    \par
    For a policy, if for each $v_n(a|h_n)$, we select action $a$ w.p.1, then we call it
    a determined policy, all determined polices is denoted as $\Pi^d$.
    \par
    For a policy, if each $v_n(a|h_n)$ only related to initial state $i_0$ and state
    of time $n$ $i_n$, i.e., for any $n$, we have $v_n(a|h_n) = v_n(a|i_0, i_n)$,
    then we call it a semi-Markov policy, denoted as $\Pi_{sm}$.
    Similarily, we can define Markov policy $\Pi_m$.
    \par
    For a Markov policy, if it is also determined policy, and $v_n(a|i_n) = v(a|i)$,
    then we call it determined steady Markov policy $\Pi_s^d$,
    which is a mapping from state space to action space, i.e., $v: \Phi \rightarrow A$.
    \par
    If not specified, we only need to find the optimal policy in the determined steady
    policy set.

    \subsection{Performance evaluation}
    \subsubsection{Performance evaluations of DTMDP}
    DTMDP performance evaluations can be devided into the following three classes.
    \begin{align}
        \eta_N^v(i) = \sum_{n=0}^N E_v \{ f(X_n, v(X_n)) | X_0 = i \}, i \in \Phi
    \end{align}
    where $E_v$ means the expectation over policy $v \in \Pi$. This is called finite
    time performance evaluation.
    \begin{align}
        \eta_\alpha^v(i) = \sum_{n=0}^\infty E_v \{ \alpha^n f(X_n, v(X_n)) | X_0 = i \}, i \in \Phi
    \end{align}
    where $\alpha$ is called the discount factor.

    \end{document}